\documentclass[12pt, a4paper, twoside]{report}
\usepackage{pgfplots}
\usepackage[utf8]{inputenc}
\usepackage{hyphenat}
\usepackage{indentfirst}

\hyphenpenalty 10000
\exhyphenpenalty 10000
\pgfplotsset{compat = newest}


\title{Real Time Embedded Systems-1st assignment}
\author{Erika Koro}
\date{March 13, 2022}



\begin{document}

    \maketitle

    \newpage 
    \section*{Objective}

    The objective of this assignment is to count the time between the insertion of an object
    in the FIFO queue by the "producer" thread and its extraction by the "consumer" thread without
    executing the function pointer.For a big number of loops,the optimal number of consumer-threads,
    that minimizes the average waiting time, needs to be found as well.

    \section*{Experiments}
    
    The experiments were executed in an AMD Ryzen 7 5800H, 3200.000 MHz. 
    In order to stabilize the measurements, the number of loops was set at 100000.


    \section*{Observations}

    In the beginning the number of consumer-threads was kept stable and the number of producer-threads
    was given different values. While the number of consumer-threads was bigger than the number of producer-threads
    the average waiting time was very big. Keeping the producer-threads stable and giving bigger numbers
    to the consumer-threads an optimization of the average waiting time was observed. What needs to be
    underlined is that the best average waiting time is notified for a number of consumer-threads that is
    16 times bigger than the number of producer-threads.   

    \newpage
    \section*{Plots}

    Producer-Threads = 16(stable).
    \pgfplotstableread{RTESplots.dat}{\table}

    \begin{tikzpicture}
        \begin{axis}[
            title style={anchor=north, yshift=20},
            title = {$Consumers-Waiting\ time$},
            xmin = 0, xmax = 256,
            ymin = 0, ymax = 25,
            xtick distance = 32,
            ytick distance = 5,
            grid = both,
            minor tick num = 1,
            major grid style = {lightgray},
            minor grid style = {lightgray!25},
            width = \textwidth,
            height = 0.7\textwidth,
            xlabel = {$consumer-threads$},
            ylabel = {$average\ waiting\ time$}
        ]

        \addplot[magenta, mark = *] table [x = {x}, y = {y}]{\table};
        
        \legend{
            Plot with marks and line
        }
        \end{axis}
    \end{tikzpicture}

    \section*{Conclusion}

    To sum up, by increasing the number of producer-threads the FIFO queue, most of the times, is full
    and as a result the average waiting increases. On the contrast, by keeping the number of 
    consumer-threads 16 times bigger than the number of producer-threads makes the FIFO queue, most
    of the times, empty and so the waiting time decreases.

\end{document}
